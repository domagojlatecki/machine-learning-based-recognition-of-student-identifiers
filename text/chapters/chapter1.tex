\chapter{Uvod}
\label{ch:uvod}
Raspoznavanja teksta je problem koji se već otprilike istražuje zadnjih 150 godina i za koji postoji širok skup metoda i
pristupa. Pregled područja raspoznavanja teksta dostupan je u\ \ref{ch:pregled-podrucja}.\ poglavlju. Cilj ovog rada bio
je koristiti jedan od pristupa raspoznavanju rukom pisanih znamenaka koristeći neki od algoritama strojnog učenja te
analizirati slučajeve u kojima odabrani algoritam pogrešno raspoznaje znamenke. Specifičan problem kojim se
bavi ovaj rad je raspoznavanje rukom pisanih studentskih identifikacijskih brojeva koji je opisan
u\ \ref{ch:opis-problema}.\ poglavlju. Navedeno poglavlje također sadrži opis odabranog pristupa raspoznavanju. Kao
odabrani model strojnog učenja korištena je unaprijedna neuronska mreža.
\ref{ch:koristeni-modeli-strojnog-ucenja}.\ poglavlje sadrži opis korištenog modela neuronske mreže i algoritama
gradijentnog spusta te su navedeni matematički izvodi. Konačno, u\ \ref{ch:rezultati-i-analiza}.\ poglavlju opisani su
postignuti rezultati i kratka analiza pogrešnih klasifikacija pri raspoznavanju znamenaka. Također je opisan skupljeni
skup slika za učenje i ispitivanje rezultata neuronske mreže.\\
Izvorni programski kod korišten u ovom radu dostupan je na sljedećoj \emph{web} adresi:\\
\small\href{https://github.com/domagojlatecki/machine-learning-based-recognition-of-student-identifiers}
{\texttt{https://github.com/domagojlatecki/machine-learning-based-\\recognition-of-student-identifiers}}\\
\normalsize
Struktura projekta na navedenoj adresi opisana je u
dodatku\ \ref{ch:opis-strukture-projekta-i-koristenih-programskih-alata}. Za implementaciju cijelog sustava za
raspoznavanje korišten je programski jezik \emph{Scala}. Detaljan opis implementacije neuronske mreže koristeći
programski jezik \emph{Scala} dostupan je u
dodatku\ \ref{ch:implementacija-neuronske-mreze-i-gradijentnog-spusta-u-programskom-jeziku-scala}. Skupljeni skup
podataka dostupan je na \emph{web} adresi:\\
\small
\url{https://github.com/domagojlatecki/handwritten-number-dataset}
\normalsize

\chapter{Zaključak}
\label{ch:zakljucak}
Raspoznavanje rukom pisanih brojeva je problem koji je već značajno istražen i postoje mnogi pristupi koji nude razne
vrijednosti točnosti klasifikacije. Prije samog postupka klasifikacije potrebno je obraditi ulazne slike tako da se
provede binarizacija slike koja znatno olakšava sljedeće faze raspoznavanja znamenaka. Potrebno je posvetiti posebnu
pažnju na postupak segmentacije jer on značajno utječe na rezultate klasifikacije jer se greške kod segmentacije šalju
dalje kroz klasifikator. Stoga se preporučuje koristiti neki od algoritama strojnog učenja umjesto determinističkog
algoritma za segmentaciju. Za postupak klasifikacije potrebno je odabrati značajke koje dovoljno dobro opisuju razlike
među pojedinim brojevima. Čak i male sličnosti među značajkama dvije različite znamenke mogu prouzročiti greške u
klasifikaciji. Unaprijedna neuronska mreža pokazala se moćnim alatom strojnog učenja koji omogućuje učenje gradijentnim
spustom. Neuronska mreža korištena u ovom radu također nudi vjerojatnosnu interpretaciju izlaznih vrijednosti
klasifikacije, pa se stoga pomoću nje mogu sagraditi napredniji sustavi koji koriste izlaze neuronske mreže kao povratnu
informaciju pri postupku segmentacije. U tom slučaju se pri postupku segmentacije pokušava maksimizirati vjerojatnost
svake pojedine znamenke što će uzrokovati boje rezultate klasifikacije.
